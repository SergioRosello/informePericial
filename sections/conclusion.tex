Tras el análisis de la imagen del sistema proporcionado, y teniendo en cuenta las exigencias legislativas, el perito concluye que hay evidencias de un ataque por parte de la persona detrás de la dirección \gls{IP} 10.0.9.6 puesto que existe tráfico generado con herramientas de enumeración de puertos y servicios como \gls{nmap}. 

Además, hay evidencias de una comunicación via correo electrónico con dos cuentas:
\begin{itemize}
\item{\textbf{rwesleay@}: cuyo ``Subject'' es: ``Re: Tarea pociones'' en ambos registros}
\item{\textbf{Hermione G}: cuyo ``Subject'' es: ``Re: Dudas del trabajo'' en un registro y ``Re: Tarea'' en el segundo registro}
\end{itemize}

La línea temporal de eventos, sugiere que el usuario llamado Snape en el equipo spape-pc ha descargado y ejecutado un programa llamado TAREA.EXE, que al ejecutarse, le ha pedido permisos de administración del sistema y el usuario ha concedido mientras estaba usando el cliente de correo electrónico Thunderbird.

Este archivo ejecutable malicioso es el que ha permitido al atacante entrar en el sistema mediante la técnica de namedpipe.

El perito no ha encontrado evidencias de que se haya entrado en la cuenta de Snape desde el ordenador del atacante hasta las 10:10:52, en este momento, el archivo de las notas de las pociones ``NOTAS POCIONES.ods'' ya se ha borrado del sistema. Tampoco se han encontrado evidencias de que dicho archivo se haya borrado del sistema.

Siguiendo la línea temporal, vemos que el usuario atacante entra en la cuenta de Snape, crea una cuenta nueva: ``Slytherin'' y la configura, entra en ella para posteriormente salir y volver a entrar en la de Snape.

Más adelante, a las 10:31:50 (UTC+0) se conecta por \gls{RDP} desde su IP ``10.0.9.6'' a la cuenta del Profesor Snape.

En este momento, abre Firefox, el navegador que usa el usuario Snape y entra al portal Moodle, donde están almacenadas las notas de los alumnos. 

Dado que el Profesor Snape usa ese navegador a diario, tiene las cookies configuradas para recordar su nombre de usuario (Con privilegios suficientes para añadir notas, cambiar notas y borrar notas).

El atacante procede a cambiar las notas y posteriormente cierra sesión en la plataforma Moodle y apaga el equipo.

\subsubsection{En resumen:}

\begin{itemize}
\item{\textbf{¿Quién lo hizo?}: El usuario manejando el sistema detrás de la IP 10.0.9.6}
\item{\textbf{¿Qué ha sucedido?}: El usuario Snape ha descargado un programa malicioso que ha permitido al atacante obtener permisos de administración en el sistema y configurar una sesión RDP para cambiar las notas en moodle.}
\item{\textbf{¿Dónde ha sucedido?}: En la cuenta del Profesor Snape, en el equipo snape-pc con IP 10.0.9.5}
\item{\textbf{¿Cuando ha sucedido?}: El atacante ha empezado a cambiar las notas a partir de las 10:31:41 (UTC+0)}
\item{\textbf{¿Como ha podido suceder?}: Ha podido cambiar las notas en Moodle porque la cuenta de Snape tenía las cookies guardadas}
\end{itemize}
